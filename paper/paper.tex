% Options for packages loaded elsewhere
\PassOptionsToPackage{unicode}{hyperref}
\PassOptionsToPackage{hyphens}{url}
\PassOptionsToPackage{dvipsnames,svgnames,x11names}{xcolor}
%
\documentclass[
]{article}

\usepackage{amsmath,amssymb}
\usepackage{lmodern}
\usepackage{iftex}
\ifPDFTeX
  \usepackage[T1]{fontenc}
  \usepackage[utf8]{inputenc}
  \usepackage{textcomp} % provide euro and other symbols
\else % if luatex or xetex
  \usepackage{unicode-math}
  \defaultfontfeatures{Scale=MatchLowercase}
  \defaultfontfeatures[\rmfamily]{Ligatures=TeX,Scale=1}
  \setmathfont[]{Latin Modern Math}
\fi
% Use upquote if available, for straight quotes in verbatim environments
\IfFileExists{upquote.sty}{\usepackage{upquote}}{}
\IfFileExists{microtype.sty}{% use microtype if available
  \usepackage[]{microtype}
  \UseMicrotypeSet[protrusion]{basicmath} % disable protrusion for tt fonts
}{}
\makeatletter
\@ifundefined{KOMAClassName}{% if non-KOMA class
  \IfFileExists{parskip.sty}{%
    \usepackage{parskip}
  }{% else
    \setlength{\parindent}{0pt}
    \setlength{\parskip}{6pt plus 2pt minus 1pt}}
}{% if KOMA class
  \KOMAoptions{parskip=half}}
\makeatother
\usepackage{xcolor}
\setlength{\emergencystretch}{3em} % prevent overfull lines
\setcounter{secnumdepth}{5}
% Make \paragraph and \subparagraph free-standing
\ifx\paragraph\undefined\else
  \let\oldparagraph\paragraph
  \renewcommand{\paragraph}[1]{\oldparagraph{#1}\mbox{}}
\fi
\ifx\subparagraph\undefined\else
  \let\oldsubparagraph\subparagraph
  \renewcommand{\subparagraph}[1]{\oldsubparagraph{#1}\mbox{}}
\fi


\providecommand{\tightlist}{%
  \setlength{\itemsep}{0pt}\setlength{\parskip}{0pt}}\usepackage{longtable,booktabs,array}
\usepackage{calc} % for calculating minipage widths
% Correct order of tables after \paragraph or \subparagraph
\usepackage{etoolbox}
\makeatletter
\patchcmd\longtable{\par}{\if@noskipsec\mbox{}\fi\par}{}{}
\makeatother
% Allow footnotes in longtable head/foot
\IfFileExists{footnotehyper.sty}{\usepackage{footnotehyper}}{\usepackage{footnote}}
\makesavenoteenv{longtable}
\usepackage{graphicx}
\makeatletter
\def\maxwidth{\ifdim\Gin@nat@width>\linewidth\linewidth\else\Gin@nat@width\fi}
\def\maxheight{\ifdim\Gin@nat@height>\textheight\textheight\else\Gin@nat@height\fi}
\makeatother
% Scale images if necessary, so that they will not overflow the page
% margins by default, and it is still possible to overwrite the defaults
% using explicit options in \includegraphics[width, height, ...]{}
\setkeys{Gin}{width=\maxwidth,height=\maxheight,keepaspectratio}
% Set default figure placement to htbp
\makeatletter
\def\fps@figure{htbp}
\makeatother
\newlength{\cslhangindent}
\setlength{\cslhangindent}{1.5em}
\newlength{\csllabelwidth}
\setlength{\csllabelwidth}{3em}
\newlength{\cslentryspacingunit} % times entry-spacing
\setlength{\cslentryspacingunit}{\parskip}
\newenvironment{CSLReferences}[2] % #1 hanging-ident, #2 entry spacing
 {% don't indent paragraphs
  \setlength{\parindent}{0pt}
  % turn on hanging indent if param 1 is 1
  \ifodd #1
  \let\oldpar\par
  \def\par{\hangindent=\cslhangindent\oldpar}
  \fi
  % set entry spacing
  \setlength{\parskip}{#2\cslentryspacingunit}
 }%
 {}
\usepackage{calc}
\newcommand{\CSLBlock}[1]{#1\hfill\break}
\newcommand{\CSLLeftMargin}[1]{\parbox[t]{\csllabelwidth}{#1}}
\newcommand{\CSLRightInline}[1]{\parbox[t]{\linewidth - \csllabelwidth}{#1}\break}
\newcommand{\CSLIndent}[1]{\hspace{\cslhangindent}#1}

\usepackage{arxiv}
\usepackage{orcidlink}
\usepackage{amsmath}
\usepackage[T1]{fontenc}
\makeatletter
\makeatother
\makeatletter
\makeatother
\makeatletter
\@ifpackageloaded{caption}{}{\usepackage{caption}}
\AtBeginDocument{%
\ifdefined\contentsname
  \renewcommand*\contentsname{Table of contents}
\else
  \newcommand\contentsname{Table of contents}
\fi
\ifdefined\listfigurename
  \renewcommand*\listfigurename{List of Figures}
\else
  \newcommand\listfigurename{List of Figures}
\fi
\ifdefined\listtablename
  \renewcommand*\listtablename{List of Tables}
\else
  \newcommand\listtablename{List of Tables}
\fi
\ifdefined\figurename
  \renewcommand*\figurename{Figure}
\else
  \newcommand\figurename{Figure}
\fi
\ifdefined\tablename
  \renewcommand*\tablename{Table}
\else
  \newcommand\tablename{Table}
\fi
}
\@ifpackageloaded{float}{}{\usepackage{float}}
\floatstyle{ruled}
\@ifundefined{c@chapter}{\newfloat{codelisting}{h}{lop}}{\newfloat{codelisting}{h}{lop}[chapter]}
\floatname{codelisting}{Listing}
\newcommand*\listoflistings{\listof{codelisting}{List of Listings}}
\makeatother
\makeatletter
\@ifpackageloaded{caption}{}{\usepackage{caption}}
\@ifpackageloaded{subcaption}{}{\usepackage{subcaption}}
\makeatother
\makeatletter
\@ifpackageloaded{tcolorbox}{}{\usepackage[many]{tcolorbox}}
\makeatother
\makeatletter
\@ifundefined{shadecolor}{\definecolor{shadecolor}{rgb}{.97, .97, .97}}
\makeatother
\makeatletter
\makeatother
\ifLuaTeX
  \usepackage{selnolig}  % disable illegal ligatures
\fi
\IfFileExists{bookmark.sty}{\usepackage{bookmark}}{\usepackage{hyperref}}
\IfFileExists{xurl.sty}{\usepackage{xurl}}{} % add URL line breaks if available
\urlstyle{same} % disable monospaced font for URLs
\hypersetup{
  pdftitle={Demo arXiv template},
  pdfauthor={H. Sherry Zhang; Collaborators},
  pdfkeywords={spatio-temporal data, indices, data pipeline},
  colorlinks=true,
  linkcolor={blue},
  filecolor={Maroon},
  citecolor={Blue},
  urlcolor={Blue},
  pdfcreator={LaTeX via pandoc}}

\title{Demo arXiv template}
\author{
\textbf{H. Sherry Zhang}~\orcidlink{0000-0002-7122-1463}\\Department of
Econometrics and Business Statistics\\Monash University\\Melbourne,
VIC\\\href{mailto:huize.zhang@monash.edu}{huize.zhang@monash.edu}\\\\\\
\textbf{Collaborators}\\Department of Econometrics and Business
Statistics\\Monash University\\Melbourne, VIC\\}
\date{}
\begin{document}
\maketitle
\begin{abstract}
\begin{itemize}
\tightlist
\item
  Indices, useful, quantify severity, early monitoring,
\item
  A huge number of indices have been proposed by domain experts,
  however, a large majority of them are not being adopted, reused, and
  compared in research or in practice.
\item
  One of the reasons for this is the plenty of indices are quite complex
  and there is no obvious easy-to-use implementation to apply them to
  user's data.
\item
  The paper describes a general pipeline framework to construct indices
  from spatio-temporal data,
\item
  This allows all the indices to be constructed through a uniform data
  pipeline and different indices to vary on the details of each step in
  the data pipeline and their orders.
\item
  The pipeline proposed aim to smooth the workflow of index construction
  through breaking down the complicated steps proposed by various
  indices into small building blocks shared by most of the indices.
\item
  The framework will be demonstrated with drought indices as examples,
  but appliable in general to environmental indices constructed from
  multivariate spatio-temporal data
\end{itemize}
\end{abstract}
{\bfseries \emph Keywords}
\def\sep{\textbullet\ }
spatio-temporal data \sep indices \sep 
data pipeline

\ifdefined\Shaded\renewenvironment{Shaded}{\begin{tcolorbox}[breakable, enhanced, frame hidden, interior hidden, boxrule=0pt, sharp corners, borderline west={3pt}{0pt}{shadecolor}]}{\end{tcolorbox}}\fi

\begin{itemize}
\tightlist
\item
  indices, why do we need them, decision making
\item
  define what is an index, what is not even if they are called index
\item
  only very brief on indices in the introduction
\item
  potential audience + what do you think they should learn from this
  work.
\end{itemize}

\hypertarget{introduction}{%
\section{Introduction}\label{introduction}}

\textbf{Why index is useful, why people care about indices}

\begin{itemize}
\tightlist
\item
  decision making: need to consider multi-facet for allocating
  resources, {[}\ldots{]}
\item
  to quantify concept without direct measure. 1) almost impossible to
  include all the items in the market, indices created from a basket of
  representative items are useful to characterise the market condition;
  2) the concept of interest can't be directly measured,
  i.e.~livability, human developement etc
\end{itemize}

Integrate multiple information to produce a single number summary
(across time).

\textbf{Define what is an index, what is not}

Indices quantify a concept of interest that is hard to measure using
{[}\ldots{]}. In finance and economic, the behavior of stock market,
indices like Dow Jones Industrial Average, S\&P 500, and Nasdaq
Composite uses a representative set of companies. Natural hazard are
also indices difficult to define because \ldots. Various drought, flood
indices have been developed to characterise the severity and monitor
from the perspective of meteorology, agricultural, hydrology, and also
socio-economy. Examples of indices are ubiquitous: economic indices for
quantifying productibility (GDP?), inflation (CPI), {[}\ldots{]}; Social
development indices like Human Development Index and Global Liveability
Index; diversity indices to characterise the richness, evenness and
dominance in ecology

Despite many measurements are called indices, they are actually not
indices. A direct rate or transformation on a measured variable: birth
\& death rate computes the total number of new-borns and death per 1000
people. Many remote sensing measures are not indices, i.e.~Normalized
Difference Vegetation Index (NDVI), despite being named as an index, is
a mere statistical transformation: the difference of near-infrared (NIR)
and red visible channel over its sum.

\begin{itemize}
\tightlist
\item
  need think through:

  \begin{itemize}
  \tightlist
  \item
    is cash rate/ interest rate indices? employment rate? eocnomic
    indices?
  \end{itemize}
\end{itemize}

\textbf{What is the problem with current workflow on index construction}

\begin{itemize}
\item
  most indices are presented as calculated numerical or categorical
  values without open source code accompanied.
\item
  This causes the problem that if analysts are not agreed with a certain
  step in the index construction, there is no room for experiment with
  other possibility.
\end{itemize}

\textbf{who would benefit from this paper}

\begin{itemize}
\tightlist
\item
  domain scientists who propose new indices and {[}use multiple indices
  to compare{]}
\item
  decision makers, towards understand the
\item
  statisticians, so as to {[}draw focus/ advocate{]} on statistical
  workflow
\end{itemize}

\hypertarget{data-pipeline-in-r}{%
\section{Data pipeline in R}\label{data-pipeline-in-r}}

\hypertarget{tidy-data}{%
\subsection{Tidy data}\label{tidy-data}}

Before the concept of tidy data (Wickham 2014), tabular data arrive at
data analysts in all different ways. Different analysts would write
customised scripts for analysing the specific data. These scripts can be
extended to other data analysed by the same people or group but this is
not generalizable directly to another dataset. When the tidy data
concept comes, variables and values are arranged so that 1) Each
variable forms a column, 2) Each observation forms a row, and; 3) Each
type of observational unit forms a table. With this specific layout,
wrangling on tabular data can be standardised into a grammar of data
manipulation in \texttt{dplyr} (Wickham et al. 2022).

A similar issue happens with index construction where researchers
construct their own indices in their own ways and there has not yet been
a tidy principle on index construction. Also, this tidy principle on
index construction is more complex than those in tidy data and the
\texttt{dplyr} package. It has to encompass the workflow of
transformation from the raw data towards the final index series.

\hypertarget{data-pipeline}{%
\subsection{Data pipeline}\label{data-pipeline}}

Constructing a pipeline that divides a complex procedure into steps that
can be concatenated has been adopted widely in the R community.

The data pipeline in interactive graphics is a set of steps that
transform the raw data to the plots displayed on the screen. The initial
pipeline proposed by Buja et al. (1988) involves the following steps:
non-linear transformation, variables standardization, randomization,
projection engine, and viewporting. The initial pipeline proposed by
Buja et al. (1988) involves the following steps: non-linear
transformation, variables standardization, randomization, projection
engine, and viewporting. Another example in the early work of pipeline
by Sutherland et al. (2000) describes a three-step pipeline: variable
standardization, dimension reduction, and scaling data into the viewing
window. This pipeline also includes the transformation on spatial and
temporal variables, i.e.~computing time lag on temporal variables. This
pipeline also includes the transformation on spatial and temporal
variables, i.e.~computing time lag on temporal variables. Wickham et al.
(2009) argues that whether made explicit or not, pipeline has to be
presented in every graphics program and breaking down graphic rendering
into steps is also beneficial for understand the implementation and
compare between different graphic systems.

The data pipeline concept is further enhanced by the pipe operator
(\texttt{\%\textgreater{}\%}) in R where a set of operations, or steps,
can be chained together to form a set of instructions.

A more recent data pipeline is tidymodels (Kuhn and Wickham 2020), a set
of packages for machine learning models following the tidyverse
principles (Wickham et al. 2019). {[}expand on tidymodels{]}

\hypertarget{a-pipeline-for-building-natural-hazard-indices}{%
\section{A pipeline for building natural hazard
indices}\label{a-pipeline-for-building-natural-hazard-indices}}

The construction of natural hazard indices also follows a set of steps,
which is usually illustrated using a flowchart in the paper. However,
every researcher follows a certain design philosophy and steps taken in
the index constructed by different researchers are not aligned. This
discourages experiment with multiple indices. Initiate a new workflow
when computing a new index.

The most popular indices (i.e.~SPI, SPEI, etc) have existing software
implementation (\texttt{SPEI}) to be applied to a different set of data.

constructing time series index should also be encapsulated in my
framework

reusable

\hypertarget{raw-data}{%
\subsection{Raw data}\label{raw-data}}

\textbf{Another section on original data directly downloaded, can have
different spatial resolution, temporal granularity, data quality
problem. After processing them and align them together they become the
``raw data''}

The data used to construct the natural hazard index usually have three
dimensions, one for location, one for time, and one for multivariate.
Mathematically, it can be written as \(X_{j, s, t}\), where
\(j = 1, 2, \cdots, J\) for variable, \(s = 1, 2, \cdots, S\) for
location, and \(t = 1, 2, \cdots, T\) for time.

The location \(s\) can refer to vector points or areas characterised by
longitude-latitude coordinates, or raster cells obtained from satellite
images.

The time dimension \(t\) can be daily, weekly, biweekly (14-16 days),
monthly, or even quarterly

Variables

This multidimensional array structure is commonly used in geospatial
analysis

Given the variety of data sources at different spatial resolution and
temporal granularity, the raw data may first come in multiple pieces.
Sometimes, even a considerable amount of work is needed to align the
spatial and temporal extent of multivariate data.

A notation for different variables have different spatial and temporal
granularity \(X_{j_1, s_1, t_1}\)???

\hypertarget{spatial-aggregation}{%
\subsection{Spatial aggregation}\label{spatial-aggregation}}

mostly happen with raster data

\hypertarget{scaling}{%
\subsection{Scaling}\label{scaling}}

A specific transformation on the scale of the data

z-score standardising, min-max standardisation into {[}0, 1{]} or {[}0,
100{]}, percentage change on the baseline close to variable
transformation step

\hypertarget{variable-transformation}{%
\subsection{Variable transformation}\label{variable-transformation}}

Restrict it to single variable, square root, log etc could be linearly,
also non-linear

GAM, can you do additive model pairwise/ three-way

\hypertarget{temporal-processing}{%
\subsection{Temporal processing}\label{temporal-processing}}

\hypertarget{dimension-reduction}{%
\subsection{Dimension reduction}\label{dimension-reduction}}

sometimes called feature extraction in the machine learning community
With drought indices, the extraction of meaningful variables from the
original data is usually supported by the water balance model, for
example, in SPEI, the step that create \(d\) out of precipitation and
potential evapotranspiration (PET) has theoretical backup from {[}see
paper.{]}

Also include weighting

\hypertarget{normalising}{%
\subsection{Normalising}\label{normalising}}

The purpose of normalising is for cross-comparison. This step can get
criticism from analysts for \ldots{}

specifically for converting from a fitted distribution to normal score
via reverse CDF function, non-parametric formula, or empirical
approximation, a common step in many index: SPI, SSI, Z score. The
purpose of normalising is to convert the index into a standardised
series after all the steps for the ease of comparison.

\hypertarget{benchmarking}{%
\subsection{Benchmarking}\label{benchmarking}}

\hypertarget{simplification}{%
\subsection{Simplification}\label{simplification}}

Discretise the continuous index into a few labelled categories. For
communicating the severity of natural hazard to general public.

uniform workflow to work with index construction.

\begin{itemize}
\tightlist
\item
  illustration
\item
  math notation
\item
  benefit of the pipeline approach

  \begin{itemize}
  \tightlist
  \item
    index diagnostic
  \item
    uncertainty
  \end{itemize}
\end{itemize}

\hypertarget{incorporating-new-methods-into-the-pipeline}{%
\section{Incorporating new methods into the
pipeline}\label{incorporating-new-methods-into-the-pipeline}}

\hypertarget{examples}{%
\section{Examples}\label{examples}}

\hypertarget{constructing-standardised-precipitation-index-spi}{%
\subsection{Constructing Standardised Precipitation Index
(SPI)}\label{constructing-standardised-precipitation-index-spi}}

\begin{itemize}
\tightlist
\item
  a basic workflow and congruence with results in the \texttt{SPEI} pkg
\item
  allow multiple distribution fit
\item
  allow bootstrap uncertainty
\end{itemize}

\hypertarget{calculating-spei-with-raster-data}{%
\subsection{Calculating SPEI with raster
data}\label{calculating-spei-with-raster-data}}

\hypertarget{reference}{%
\section*{Reference}\label{reference}}
\addcontentsline{toc}{section}{Reference}

\hypertarget{refs}{}
\begin{CSLReferences}{1}{0}
\leavevmode\vadjust pre{\hypertarget{ref-buja_elements_1988}{}}%
Buja, A, D Asimov, C Hurley, and JA McDonald. 1988. {``Elements of a
Viewing Pipeline for Data Analysis.''} In \emph{Dynamic Graphics for
Statistics}, 277--308. Wadsworth, Belmont.

\leavevmode\vadjust pre{\hypertarget{ref-tidymodels}{}}%
Kuhn, Max, and Hadley Wickham. 2020. \emph{Tidymodels: A Collection of
Packages for Modeling and Machine Learning Using Tidyverse Principles.}
\url{https://www.tidymodels.org}.

\leavevmode\vadjust pre{\hypertarget{ref-sutherland_orca_2000}{}}%
Sutherland, Peter, Anthony Rossini, Thomas Lumley, Nicholas Lewin-Koh,
Julie Dickerson, Zach Cox, and Dianne Cook. 2000. {``Orca: {A}
{Visualization} {Toolkit} for {High}-{Dimensional} {Data}.''}
\emph{Journal of Computational and Graphical Statistics} 9 (3): 509--29.
\url{https://www.jstor.org/stable/1390943}.

\leavevmode\vadjust pre{\hypertarget{ref-wickham_tidy_2014}{}}%
Wickham, Hadley. 2014. {``Tidy {Data}.''} \emph{Journal of Statistical
Software} 59 (September): 1--23.
\url{https://doi.org/10.18637/jss.v059.i10}.

\leavevmode\vadjust pre{\hypertarget{ref-tidyverse}{}}%
Wickham, Hadley, Mara Averick, Jennifer Bryan, Winston Chang, Lucy
D'Agostino McGowan, Romain François, Garrett Grolemund, et al. 2019.
{``Welcome to the {tidyverse}.''} \emph{Journal of Open Source Software}
4 (43): 1686. \url{https://doi.org/10.21105/joss.01686}.

\leavevmode\vadjust pre{\hypertarget{ref-dplyr}{}}%
Wickham, Hadley, Romain François, Lionel Henry, and Kirill Müller. 2022.
\emph{Dplyr: A Grammar of Data Manipulation}.

\leavevmode\vadjust pre{\hypertarget{ref-wickham_plumbing_2009}{}}%
Wickham, Hadley, Michael Lawrence, Dianne Cook, Andreas Buja, Heike
Hofmann, and Deborah F. Swayne. 2009. {``The Plumbing of Interactive
Graphics.''} \emph{Computational Statistics} 24 (2): 207--15.
\url{https://doi.org/10.1007/s00180-008-0116-x}.

\end{CSLReferences}



\end{document}
