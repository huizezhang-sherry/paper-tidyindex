% Options for packages loaded elsewhere
\PassOptionsToPackage{unicode}{hyperref}
\PassOptionsToPackage{hyphens}{url}
\PassOptionsToPackage{dvipsnames,svgnames,x11names}{xcolor}
%
\documentclass[
]{interact}

\usepackage{amsmath,amssymb}
\usepackage{lmodern}
\usepackage{iftex}
\ifPDFTeX
  \usepackage[T1]{fontenc}
  \usepackage[utf8]{inputenc}
  \usepackage{textcomp} % provide euro and other symbols
\else % if luatex or xetex
  \usepackage{unicode-math}
  \defaultfontfeatures{Scale=MatchLowercase}
  \defaultfontfeatures[\rmfamily]{Ligatures=TeX,Scale=1}
\fi
% Use upquote if available, for straight quotes in verbatim environments
\IfFileExists{upquote.sty}{\usepackage{upquote}}{}
\IfFileExists{microtype.sty}{% use microtype if available
  \usepackage[]{microtype}
  \UseMicrotypeSet[protrusion]{basicmath} % disable protrusion for tt fonts
}{}
\makeatletter
\@ifundefined{KOMAClassName}{% if non-KOMA class
  \IfFileExists{parskip.sty}{%
    \usepackage{parskip}
  }{% else
    \setlength{\parindent}{0pt}
    \setlength{\parskip}{6pt plus 2pt minus 1pt}}
}{% if KOMA class
  \KOMAoptions{parskip=half}}
\makeatother
\usepackage{xcolor}
\setlength{\emergencystretch}{3em} % prevent overfull lines
\setcounter{secnumdepth}{5}
% Make \paragraph and \subparagraph free-standing
\ifx\paragraph\undefined\else
  \let\oldparagraph\paragraph
  \renewcommand{\paragraph}[1]{\oldparagraph{#1}\mbox{}}
\fi
\ifx\subparagraph\undefined\else
  \let\oldsubparagraph\subparagraph
  \renewcommand{\subparagraph}[1]{\oldsubparagraph{#1}\mbox{}}
\fi

\usepackage{color}
\usepackage{fancyvrb}
\newcommand{\VerbBar}{|}
\newcommand{\VERB}{\Verb[commandchars=\\\{\}]}
\DefineVerbatimEnvironment{Highlighting}{Verbatim}{commandchars=\\\{\}}
% Add ',fontsize=\small' for more characters per line
\usepackage{framed}
\definecolor{shadecolor}{RGB}{241,243,245}
\newenvironment{Shaded}{\begin{snugshade}}{\end{snugshade}}
\newcommand{\AlertTok}[1]{\textcolor[rgb]{0.68,0.00,0.00}{#1}}
\newcommand{\AnnotationTok}[1]{\textcolor[rgb]{0.37,0.37,0.37}{#1}}
\newcommand{\AttributeTok}[1]{\textcolor[rgb]{0.40,0.45,0.13}{#1}}
\newcommand{\BaseNTok}[1]{\textcolor[rgb]{0.68,0.00,0.00}{#1}}
\newcommand{\BuiltInTok}[1]{\textcolor[rgb]{0.00,0.23,0.31}{#1}}
\newcommand{\CharTok}[1]{\textcolor[rgb]{0.13,0.47,0.30}{#1}}
\newcommand{\CommentTok}[1]{\textcolor[rgb]{0.37,0.37,0.37}{#1}}
\newcommand{\CommentVarTok}[1]{\textcolor[rgb]{0.37,0.37,0.37}{\textit{#1}}}
\newcommand{\ConstantTok}[1]{\textcolor[rgb]{0.56,0.35,0.01}{#1}}
\newcommand{\ControlFlowTok}[1]{\textcolor[rgb]{0.00,0.23,0.31}{#1}}
\newcommand{\DataTypeTok}[1]{\textcolor[rgb]{0.68,0.00,0.00}{#1}}
\newcommand{\DecValTok}[1]{\textcolor[rgb]{0.68,0.00,0.00}{#1}}
\newcommand{\DocumentationTok}[1]{\textcolor[rgb]{0.37,0.37,0.37}{\textit{#1}}}
\newcommand{\ErrorTok}[1]{\textcolor[rgb]{0.68,0.00,0.00}{#1}}
\newcommand{\ExtensionTok}[1]{\textcolor[rgb]{0.00,0.23,0.31}{#1}}
\newcommand{\FloatTok}[1]{\textcolor[rgb]{0.68,0.00,0.00}{#1}}
\newcommand{\FunctionTok}[1]{\textcolor[rgb]{0.28,0.35,0.67}{#1}}
\newcommand{\ImportTok}[1]{\textcolor[rgb]{0.00,0.46,0.62}{#1}}
\newcommand{\InformationTok}[1]{\textcolor[rgb]{0.37,0.37,0.37}{#1}}
\newcommand{\KeywordTok}[1]{\textcolor[rgb]{0.00,0.23,0.31}{#1}}
\newcommand{\NormalTok}[1]{\textcolor[rgb]{0.00,0.23,0.31}{#1}}
\newcommand{\OperatorTok}[1]{\textcolor[rgb]{0.37,0.37,0.37}{#1}}
\newcommand{\OtherTok}[1]{\textcolor[rgb]{0.00,0.23,0.31}{#1}}
\newcommand{\PreprocessorTok}[1]{\textcolor[rgb]{0.68,0.00,0.00}{#1}}
\newcommand{\RegionMarkerTok}[1]{\textcolor[rgb]{0.00,0.23,0.31}{#1}}
\newcommand{\SpecialCharTok}[1]{\textcolor[rgb]{0.37,0.37,0.37}{#1}}
\newcommand{\SpecialStringTok}[1]{\textcolor[rgb]{0.13,0.47,0.30}{#1}}
\newcommand{\StringTok}[1]{\textcolor[rgb]{0.13,0.47,0.30}{#1}}
\newcommand{\VariableTok}[1]{\textcolor[rgb]{0.07,0.07,0.07}{#1}}
\newcommand{\VerbatimStringTok}[1]{\textcolor[rgb]{0.13,0.47,0.30}{#1}}
\newcommand{\WarningTok}[1]{\textcolor[rgb]{0.37,0.37,0.37}{\textit{#1}}}

\providecommand{\tightlist}{%
  \setlength{\itemsep}{0pt}\setlength{\parskip}{0pt}}\usepackage{longtable,booktabs,array}
\usepackage{calc} % for calculating minipage widths
% Correct order of tables after \paragraph or \subparagraph
\usepackage{etoolbox}
\makeatletter
\patchcmd\longtable{\par}{\if@noskipsec\mbox{}\fi\par}{}{}
\makeatother
% Allow footnotes in longtable head/foot
\IfFileExists{footnotehyper.sty}{\usepackage{footnotehyper}}{\usepackage{footnote}}
\makesavenoteenv{longtable}
\usepackage{graphicx}
\makeatletter
\def\maxwidth{\ifdim\Gin@nat@width>\linewidth\linewidth\else\Gin@nat@width\fi}
\def\maxheight{\ifdim\Gin@nat@height>\textheight\textheight\else\Gin@nat@height\fi}
\makeatother
% Scale images if necessary, so that they will not overflow the page
% margins by default, and it is still possible to overwrite the defaults
% using explicit options in \includegraphics[width, height, ...]{}
\setkeys{Gin}{width=\maxwidth,height=\maxheight,keepaspectratio}
% Set default figure placement to htbp
\makeatletter
\def\fps@figure{htbp}
\makeatother
\newlength{\cslhangindent}
\setlength{\cslhangindent}{1.5em}
\newlength{\csllabelwidth}
\setlength{\csllabelwidth}{3em}
\newlength{\cslentryspacingunit} % times entry-spacing
\setlength{\cslentryspacingunit}{\parskip}
\newenvironment{CSLReferences}[2] % #1 hanging-ident, #2 entry spacing
 {% don't indent paragraphs
  \setlength{\parindent}{0pt}
  % turn on hanging indent if param 1 is 1
  \ifodd #1
  \let\oldpar\par
  \def\par{\hangindent=\cslhangindent\oldpar}
  \fi
  % set entry spacing
  \setlength{\parskip}{#2\cslentryspacingunit}
 }%
 {}
\usepackage{calc}
\newcommand{\CSLBlock}[1]{#1\hfill\break}
\newcommand{\CSLLeftMargin}[1]{\parbox[t]{\csllabelwidth}{#1}}
\newcommand{\CSLRightInline}[1]{\parbox[t]{\linewidth - \csllabelwidth}{#1}\break}
\newcommand{\CSLIndent}[1]{\hspace{\cslhangindent}#1}

\usepackage{orcidlink}
\makeatletter
\makeatother
\makeatletter
\makeatother
\makeatletter
\@ifpackageloaded{caption}{}{\usepackage{caption}}
\AtBeginDocument{%
\ifdefined\contentsname
  \renewcommand*\contentsname{Table of contents}
\else
  \newcommand\contentsname{Table of contents}
\fi
\ifdefined\listfigurename
  \renewcommand*\listfigurename{List of Figures}
\else
  \newcommand\listfigurename{List of Figures}
\fi
\ifdefined\listtablename
  \renewcommand*\listtablename{List of Tables}
\else
  \newcommand\listtablename{List of Tables}
\fi
\ifdefined\figurename
  \renewcommand*\figurename{Figure}
\else
  \newcommand\figurename{Figure}
\fi
\ifdefined\tablename
  \renewcommand*\tablename{Table}
\else
  \newcommand\tablename{Table}
\fi
}
\@ifpackageloaded{float}{}{\usepackage{float}}
\floatstyle{ruled}
\@ifundefined{c@chapter}{\newfloat{codelisting}{h}{lop}}{\newfloat{codelisting}{h}{lop}[chapter]}
\floatname{codelisting}{Listing}
\newcommand*\listoflistings{\listof{codelisting}{List of Listings}}
\makeatother
\makeatletter
\@ifpackageloaded{caption}{}{\usepackage{caption}}
\@ifpackageloaded{subcaption}{}{\usepackage{subcaption}}
\makeatother
\makeatletter
\@ifpackageloaded{tcolorbox}{}{\usepackage[many]{tcolorbox}}
\makeatother
\makeatletter
\@ifundefined{shadecolor}{\definecolor{shadecolor}{rgb}{.97, .97, .97}}
\makeatother
\makeatletter
\makeatother
\ifLuaTeX
  \usepackage{selnolig}  % disable illegal ligatures
\fi
\IfFileExists{bookmark.sty}{\usepackage{bookmark}}{\usepackage{hyperref}}
\IfFileExists{xurl.sty}{\usepackage{xurl}}{} % add URL line breaks if available
\urlstyle{same} % disable monospaced font for URLs
\hypersetup{
  pdftitle={A Tidy Framework and Infrastructure to Systematically Assemble Spatio-temporal Indexes from Multivariate Data},
  pdfauthor={H. Sherry Zhang; Dianne Cook; Ursula Laa; Nicolas Langrené; Patricia Menéndez},
  pdfkeywords={indexes, data pipeline, software
design, uncertainty, decision-making},
  colorlinks=true,
  linkcolor={blue},
  filecolor={Maroon},
  citecolor={Blue},
  urlcolor={Blue},
  pdfcreator={LaTeX via pandoc}}

\title{A Tidy Framework and Infrastructure to Systematically Assemble
Spatio-temporal Indexes from Multivariate Data}
\author{H. Sherry
Zhang$\textsuperscript{1}$~\orcidlink{0000-0002-7122-1463}, Dianne
Cook$\textsuperscript{1}$~\orcidlink{0000-0002-3813-7155}, Ursula
Laa$\textsuperscript{2}$~\orcidlink{0000-0002-0249-6439}, Nicolas
Langrené$\textsuperscript{3}$~\orcidlink{0000-0001-7601-4618}, Patricia
Menéndez$\textsuperscript{1}$~\orcidlink{0000-0003-0701-6315}}

\thanks{CONTACT: H. Sherry
Zhang. Email: \href{mailto:huize.zhang@monash.edu}{\nolinkurl{huize.zhang@monash.edu}}. }
\begin{document}
\captionsetup{labelsep=space}
\maketitle
\textsuperscript{1} Department of Econometrics and Business
Statistics, Monash University, Melbourne,
Victoria, Australia\\ \textsuperscript{2} Institute of
Statistics, University of Natural Resources and Life
Sciences, Vienna, Austria\\ \textsuperscript{3} Department of
Mathematical Sciences, BNU-HKBU United International College, Zhuhai,
Guangdong, China
\begin{abstract}
Indexes are useful for summarizing multivariate information into single
metrics for monitoring, communicating, and decision-making. While most
work has focused on defining new indexes for specific purposes, more
attention needs to be directed towards making it possible to understand
index behavior in different data conditions, and to determine how their
structure affects their value and variation in values. Here we discuss a
modular data pipeline recommendation to assemble indexes. It is
universally applicable to index computation and allows investigation of
index behavior as part of the development procedure. One can compute
with different the index with different parameter choices, adjust steps
in the index definition by adding, removing, and swapping them to
experiment with various index designs, calculate uncertainty measures,
and assess an index's robustness. The paper presents three examples to
illustrate the pipeline framework usage: comparison of two different
indexes designed to monitor the spatio-temporal distribution of drought
in Queensland, Australia; the effect of dimension reduction choices on
the Global Gender Gap Index (GGGI) on a country's ranking; and how to
calculate bootstrap confidence intervals for XXX index. The methods are
supported by a new R package, called \texttt{tidyindex}.
\end{abstract}
\begin{keywords}
\def\sep{;\ }
indexes\sep data pipeline\sep software design\sep uncertainty\sep 
decision-making
\end{keywords}
\ifdefined\Shaded\renewenvironment{Shaded}{\begin{tcolorbox}[frame hidden, boxrule=0pt, interior hidden, borderline west={3pt}{0pt}{shadecolor}, sharp corners, enhanced, breakable]}{\end{tcolorbox}}\fi

\hypertarget{introduction}{%
\section{Introduction}\label{introduction}}

Indexes are commonly used to combine and summarize different sources of
information into a single number for monitoring, communicating, and
decision-making. They serve as critical tools across the natural and
social sciences. Examples include the Air Quality Index, El
Niño-Southern Oscillation Index, Consumer Price Index, QS University
Rankings, and the Human Development Index. In environmental science
climate indexes are produced by major monitoring centers, like the
United States Drought Monitor and National Oceanic and Atmospheric
Administration, to facilitate agricultural planning and early detection
of natural disasters. In economics, indexes provide insight into market
trends through combining prices of a basket of goods and services. In
social sciences, indexes are used to monitor human development, gender
equity, or university quality. The problem is that every index is
developed in its own unique way, by different researchers or
organizations, and often indexes designed for the same purpose cannot
easily be compared.

To construct an index, experts typically start by defining a concept of
interest that requires measurement. This concept often lacks a direct
measurable attribute or can only be measured as a composite of various
processes, yet it holds social and public significance. To create an
index, once the underlying processes involved are identified, relevant
and available variables are then defined, collected, and combined using
statistical methods into an index that aims to measure the process of
interest. The construction process is often not straightforward, and
decisions need to be made, such as the selection of variables to be
included, which might depend on data availability and the statistical
definition of the index to be used, among others. For instance, the
indexes constructed from a linear combination of variables need to
decide on the weight assigned to each variable. Some indexes have a
spatial and/or temporal component, and variables can be aggregated to
different spatial resolutions and temporal scales, leading to various
indexes for different monitoring purposes. Hence, all these decisions
can result in different index values and have different practical
implications.

To be able to test different decision choices systematically for an
index, the index needs to be broken down into its fundamental building
blocks to analyze the contribution and effect of each component. We call
this process of breaking the index construction into different steps the
index pipeline. Such decomposition of index components provides the
means to standardize index construction via a pipeline and offers
benefits for comparing among indexes and calculating index uncertainty.

In this work, we provide statistical and computational methods for
developing a data pipeline framework to construct and customize indexes
using data. The proposed pipeline comprises various modules, including
temporal and spatial aggregation, variable transformation and
combination, distribution fitting, benchmark setting, and index
communication. Given the decisions analysts need to make when combining
multivariate data into indexes, the proposed pipeline enables the
evaluation of how the specific choice can affect the index, as well as
how the index may appear under alternative options. Furthermore,
uncertainty calculation can also flow through the pipeline, providing
the index with confidence measures.

The rest of the paper is structured as follows: Section~\ref{sec-tidy}
reviews the tidy framework in R and how index construction can benefit
from such a framework. The details of the pipeline modules are presented
in Section~\ref{sec-pipeline}. Section~\ref{sec-software} explains the
design of the \texttt{tidyindex} package that implements the modules.
Examples are given in Section~\ref{sec-examples} to illustrate the use
cases of the pipeline.

\hypertarget{background-to-index-development}{%
\section{Background to index
development}\label{background-to-index-development}}

There are many documents providing advice on how to construct indexes
for different fields, and review articles describing the range of
available indexes for specific purposes. The OECD handbook (OECD,
European Union, and Joint Research Centre - European Commission 2008)
provides a comprehensive guide for computing socio-economic composite
indexes, with detailed steps and recommendations. The drought index
handbook (Svoboda, Fuchs, et al. 2016) provides details of various
drought indexes and recommendations from the World Meteorology
Organization. Zargar et al. (2011), Hao and Singh (2015) and Alahacoon
and Edirisinghe (2022) are review papers describing the range of
possible drought indexes.

There is also some attention being given to the diagnosis of indexes,
and incorporation of uncertainty. Jones and Andrey (2007) investigates
the methodological choices made in the development of indexes for
assessing vulnerable neighborhoods. Saisana, Saltelli, and Tarantola
(2005) describes incorporating uncertainty estimates and conducting
sensitivity analysis on composite indexes. Tate (2012), similarly, makes
a comparative assessment of social vulnerability indexes based on
uncertainty estimation and sensitivity analysis. (XXX Something about
Ursula's colleagues paper here too.)

There are also R packages supporting index calculation. The
\texttt{SPEI} package (Vicente-Serrano, Beguería, and López-Moreno 2010)
computes two drought indexes. The \texttt{gpindex} package (Martin 2023)
computes price indexes, and the \texttt{fundiversity} package (Grenié
and Gruson 2023) computes functional diversity indexes for ecological
study. The package \texttt{COINr} (Becker et al. 2022) is more
ambitious, making a start on following the broader guidelines in the
OECD handbook to construct, analyze, and visualize composite indexes.

From reviewing this literature, and in the process of developing methods
for making it easier to work with multivariate spatiotemporal data, it
seems possible to think about indexes in a more organised, cohesive and
standard manner. Actually, it seems that the area could benefit from a
\emph{tidy} approach.

\hypertarget{sec-tidy}{%
\section{Tidy framework}\label{sec-tidy}}

The tidy framework consists of two key components: tidy data and tidy
tools. The concept of tidy data (Wickham 2014) prescribes specific rules
for organizing data in an analysis, with observations as rows, variables
as columns, and types of observational units as tables. Tidy tools, on
the other hand, are concatenated in a sequence through which the tidy
data flows, creating a pipeline for data processing and modeling. These
pipelines are data-centric, meaning all the tidy tools or functions take
a tidy data object as input and return a processed tidy data object,
directly ready for the next operations to be applied. Also, the pipeline
approach corresponds to the modular programming practice, which breaks
down complex problems into smaller and more manageable pieces, as
opposed to a monolithic design, where all the steps are predetermined
and integrated into a single piece. The flexibility provided by the
modularity makes it easier to modify certain steps in the pipeline and
to maintain and extend the code base.

Examples of using a pipeline approach for data analysis can be traced
back to the interactive graphics literature, including Buja et al.
(1988); Sutherland et al. (2000); Xie, Hofmann, and Cheng (2014);
Wickham et al. (2009). Wickham et al. (2009) argue that whether made
explicit or not, a pipeline has to be presented in every graphics
program, and making them explicit is beneficial for understanding the
implementation and comparing between different graphic systems. While
this comment is made in the context of interactive graphics programs, it
is also applicable generally to any data analysis workflow. More
recently, the tidyverse suite (Wickham et al. 2019) takes the pipeline
approach for general-purpose data wrangling and has gained popularity
within the R community. The pipeline-style code can be directly read as
a series of operations applied successively on tidy data objects,
offering a method to document the data wrangling process with all the
computational details for reproducibility.

Since the success of tidyverse, more packages have been developed to
analyze data using the tidy framework for domains specific applications,
a noticeable example of which is \texttt{tidymodels} for building
machine learning models (Kuhn and Wickham 2020). To create a tidy
workflow tailored to a specific domain, developers first need to
identify the fundamental building blocks to create a workflow. These
components are then implemented as modules, which can be combined to
form the pipeline. For example, in supervised machine learning models,
steps such as data splitting, model training, and model evaluation are
commonly used in most workflow. In the \texttt{tidymodels}, these steps
are correspondingly implemented as package \texttt{rsample},
\texttt{parsnip}, and \texttt{yardstick}, agnostic to the specific model
chosen. The uniform interface in tidymodels frees analysts from
recalling model-specific syntax for performing the same operation across
different models, increasing the efficiency to work with different
models simultaneously.

For constructing indexes, the pipeline approach adopts explicit and
standalone modules that can be assembled in different ways. Index
developers can choose the appropriate modules and arrange them
accordingly to generate the data pipeline that is needed for their
purpose. The pipeline approach provides many advantages:

\begin{itemize}
\tightlist
\item
  makes the computation more transparent, and thus more easily debugged.
\item
  allows for rapidly processing new data to check how different
  features, like outliers, might affect the index value.
\item
  provides the capacity to measure uncertainty by computing confidence
  intervals from multiple samples as generated by bootstrapping to
  original data.
\item
  enables systematic comparison of surrogate indexes designed to measure
  the same phenomenon.
\item
  it may even be possible to automate diagrammatic explanations and
  documentation of the index.
\end{itemize}

Adoption of this pipeline approach would provide uniformity to the field
of index development, research, and application.

\hypertarget{sec-pipeline}{%
\section{Details of the index pipeline}\label{sec-pipeline}}

In constructing various indexes, the primary aim is to transform the
data, often multivariate, into a univariate index. Spatial and temporal
considerations are also factored into the process when observational
units and time periods are not independent. However, despite the
variations in contextual information for indexes in different fields,
the underlying statistical methodology remains consistent across diverse
domains. Each index can be represented as a series of modular
statistical operations on the data. This allows us to decompose the
index construction process into a unified pipeline workflow with a
standardized set of data processing steps to be applied across different
indexes.

An overview of the pipeline is presented in
Figure~\ref{fig-pipeline-steps}, illustrating the nine available modules
designed to obtain the index from the data. These modules include
operations for temporal and spatial aggregation, variable transformation
and combination, distribution fitting, benchmark setting, and index
communication. Analysts have the flexibility to construct indexes by
connecting modules according to their preferences.

Now, we introduce the notation used for describing pipeline modules.
Consider a multivariate spatio-temporal process,

\begin{equation}
\mathbf{x}(s;t) = \{x_1(s;t), x_2(s;t), \cdots, x_p(s;t)\} \qquad s \in D_s \subseteq \mathbb{R}^m, t \in D_t \subseteq \mathbb{R}^n 
\end{equation}

where:

\begin{itemize}
\item
  \(x_j(s, t)\) represents a variable of interest for example
  precipitation, \(j = 1, \cdots, p\) and
\item
  \(s\) represents the geographic locations in the space
  \(D_s \subseteq \mathbb{R}^m\). Examples of geographic locations
  include a collection of countries, longitude and latitude coordinates
  or regions of interest and,
\item
  \(t\) denotes the temporal order in \(D_t \subseteq \mathbb{R}^n\).
  For instance, time measurements could be recorded hourly, yearly,
  monthly, quarterly, or by season.
\end{itemize}

In what follows when geographic or temporal components of the
\(x_j(s,t)\) process are fixed we will be using suffix notation. For
example, \(x_{sj}(t)\) represents the data for a fixed location \(s\) as
a function of time \(t\). While \(x_{tj}(s)\) denotes the spatial
varying process for a fixed \(t\). An overview of the notation for
pipeline input, operation, and output is present in
Table~\ref{tbl-notation}.

\hypertarget{tbl-notation}{}
\begin{longtable}[]{@{}
  >{\raggedright\arraybackslash}p{(\columnwidth - 8\tabcolsep) * \real{0.0972}}
  >{\raggedright\arraybackslash}p{(\columnwidth - 8\tabcolsep) * \real{0.2083}}
  >{\raggedright\arraybackslash}p{(\columnwidth - 8\tabcolsep) * \real{0.1250}}
  >{\raggedright\arraybackslash}p{(\columnwidth - 8\tabcolsep) * \real{0.2083}}
  >{\raggedright\arraybackslash}p{(\columnwidth - 8\tabcolsep) * \real{0.3611}}@{}}
\caption{\label{tbl-notation}An notation overview of the input,
operation, and output of each pipeline module.}\tabularnewline
\toprule()
\begin{minipage}[b]{\linewidth}\raggedright
Section
\end{minipage} & \begin{minipage}[b]{\linewidth}\raggedright
Module
\end{minipage} & \begin{minipage}[b]{\linewidth}\raggedright
Input
\end{minipage} & \begin{minipage}[b]{\linewidth}\raggedright
Operation
\end{minipage} & \begin{minipage}[b]{\linewidth}\raggedright
Output
\end{minipage} \\
\midrule()
\endfirsthead
\toprule()
\begin{minipage}[b]{\linewidth}\raggedright
Section
\end{minipage} & \begin{minipage}[b]{\linewidth}\raggedright
Module
\end{minipage} & \begin{minipage}[b]{\linewidth}\raggedright
Input
\end{minipage} & \begin{minipage}[b]{\linewidth}\raggedright
Operation
\end{minipage} & \begin{minipage}[b]{\linewidth}\raggedright
Output
\end{minipage} \\
\midrule()
\endhead
3.1 & Temporal processing & \(x_{sj}(t)\) & \(f[x_{sj}(t)]\) &
\(x^{\text{Temp}}_{sj}(t^\prime) \quad t^\prime \in D_{t^\prime}\) \\
3.2 & Spatial processing & \(x_{tj}(s)\) & \(g[x_{tj}(s)]\) &
\(x^{\text{Spat}}_{tj}(s^\prime) \quad s^\prime \in D_{s^\prime}\) \\
3.3 & Variable transformation & \(x_{j}(s; t)\) & \(T[x_j(s;t)]\) &
\(x^{\text{Trans}}_j(s;t)\) \\
3.4 & Scaling & \(x_j(s; t)\) & \([x_j(s;t) - \alpha]/\gamma\) &
\(x^{\text{Scale}}_j(s;t)\) \\
3.5 & Dimension reduction & \(\mathbf{x}(s;t)\) & \(h[\mathbf{x}(s;t)]\)
& \(\mathbf{y}(s;t) \quad \mathbf{y} \subseteq \mathbb{R}^d, d < q\) \\
3.6 & Distribution fit & \(x_j(s; t)\) & \(F[x_j(s;t)]\) &
\(p_j(s;t) \quad p(.) \in [0, 1]\) \\
3.7 & Normalising & \(x_j(s; t)\) & \(\Phi^{-1}[x_j(s; t)]\) &
\(z_j(s; t)\) \\
3.8 & Benchmarking & \(x_j(s; t)\) & \(u[x_j(s;t)]\) & \(b_j(s;t)\) \\
3.9 & Simplification & \(x_j(s; t)\) & \(v[x_j(s;t)]\) &
\(A_j(s;t) \in \{a_1, a_2, \cdots, a_j\}\) \\
\bottomrule()
\end{longtable}

\begin{figure}

{\centering \includegraphics[width=1\textwidth,height=0.9\textheight]{figures/pipeline-overall.png}

}

\caption{\label{fig-pipeline-steps}Diagram of pipeline modules for index
construction. The highlighted path illustrates one possible construction
using the dimension reduction and simplification module.}

\end{figure}

\hypertarget{temporal-processing}{%
\subsection{Temporal processing}\label{temporal-processing}}

The temporal processing module takes as input argument a single variable
\(x_{sj}(t)\) at location \(s\) as a function of time. In this step the
original time series can be transformed or summarized into a new one via
time aggregation, the transformation is represented by the function
\(f\), \(x^{\text{Temp}}_{sj}(t^\prime) = f[x_{sj}(t)]\) where
\(t^\prime\) refers to the new temporal resolution after aggregation. An
example of temporal processing done in the computation of the
Standardized Precipitation Index (SPI) (McKee et al. 1993), consists of
summing the monthly precipitation series over a rolling time window of
size \(k\). That is also known as the time scale. For SPI, the choice of
the time scale \(k\) is used to control the accumulation period for the
water deficit, enabling the assessment of drought severity across
various types (meteorological, agricultural, and hydrological).

\hypertarget{spatial-processing}{%
\subsection{Spatial processing}\label{spatial-processing}}

The spatial processing module takes a single variable with a fixed
temporal dimension, \(x_{tj}(s)\), as input. This step transforms the
variable from the original spatial dimension \(s\) into the new
dimension \(s^\prime \in D_{s^\prime}\) through
\(x^{\text{Spat}}_{tj}(s^\prime) = g[x_{tj}(s)]\). The change of spatial
dimension allows for the alignment of variables collected from different
measurements, such as in-situ stations and satellite imagery, or
originating from different resolutions. This also includes the
aggregation of variables into different levels, such as city, state, and
country scales.

\hypertarget{variable-transformation}{%
\subsection{Variable transformation}\label{variable-transformation}}

Variable transformation takes the input of a single variable
\(x_j(s;t)\) and reshapes its distribution using the function \(T[.]\)
to produce \(x^{\text{Trans}}_{j}(s;t)\). When a variable has a skewed
distribution, transformations such as log, square root, or cubic root
can adjust the distribution towards normality. For example, in Human
Development Index (HDI), a logarithmic transformation is applied to the
variable Gross National Income per capita (GNI), to reduce its impact on
HDI, particularly for countries with high GNI values.

\hypertarget{scaling}{%
\subsection{Scaling}\label{scaling}}

Unlike variable transformation, scaling maintains the distributional
shape of the variable. It includes techniques such as centering, z-score
standardization, and min-max standardization and can be expressed as
\([x_{j}(s;t) - \alpha]/\gamma\). In Human Development Index (HDI), the
three dimensions (health, education, and economy) are converted into the
same scale (0-1) using min-max standardization.

\begin{figure}

{\centering \includegraphics{tidyindex_files/figure-pdf/fig-scale-var-trans-compare-1.pdf}

}

\caption{\label{fig-scale-var-trans-compare}Comparison of the module
scaling (green) and variable transformation (orange). While both modules
change the variable range, scaling maintains the same distributional
shape, which is not the case with variable transformation.}

\end{figure}

\hypertarget{dimension-reduction}{%
\subsection{Dimension reduction}\label{dimension-reduction}}

Dimension reduction takes the multivariate information
\(\mathbf{x}(s;t)\), where
\(\mathbf{x} \subseteq \mathbb{R}^q,\; q\leq p\), as the input. It
summarises the high-dimensional information into a lower-dimension
representation \(\mathbf{y}(s;t)\), where \(y \subseteq \mathbb{R}^d\)
and \(d < q\), as the output. The transformation can be based on
domain-specific knowledge, originating from theories describing the
underlying physical processes, or guided by statistical methods. For
example, the Standardized Precipitation-Evapotranspiration Index (SPEI)
(Beguería and Vicente-Serrano 2017) calculates the difference \(D\)
between precipitation (\(P\)) and potential evapotranspiration
(\(\text{PET}\)), using a water balance model (\(D = P - \text{PET}\)).
This is the only step that differs from the Standardized Precipitation
Index (SPI).

\hypertarget{distribution-fit}{%
\subsection{Distribution fit}\label{distribution-fit}}

Distribution fit applies the Cumulative Distribution Function (CDF)
\(F\) of a distribution on the variable \(x_j(s; t)\) to obtain the
probability values \(p_j(s;t) \in [0, 1]\). In SPEI, many distributions,
including log-logistic, Pearson III, lognormal, and general extreme
distribution, are candidates for the aggregated series. Different
fitting methods and different goodness of fit tests may be used to
compare the distribution choice on the index value.

\hypertarget{normalising}{%
\subsection{Normalising}\label{normalising}}

Normalizing applies the inverse normal CDF \(\Phi^{-1}\) on the input
data to obtain the normal density \(z_{j}(s;t)\). Normalizing can
sometimes be confused with the scaling or variable transformation
module, which does not involve using a normal distribution to transform
the variable. It is arguably whether normalizing and distribution fit
should be combined or separated into two modules. A decision has been
made to separate them into two modules given the different types of
output each module presents (probability values for distribution fit and
normal density values for normalizing).

\hypertarget{benchmarking}{%
\subsection{Benchmarking}\label{benchmarking}}

Benchmark sets a value \(b_j(s,t)\) for comparing against the original
variable \(x_j(s;t)\). This benchmark can be a fixed value consistently
across space and time or determined by the data through the function
\(u[x_j(s;t)]\). Once a benchmark is set, observations can be
highlighted for adjustments in other modules or can serve as targets for
monitoring and planning.

\hypertarget{simplification}{%
\subsection{Simplification}\label{simplification}}

Simplification takes a continuous variable \(x_j(s;t)\) and categorises
it into a discrete set \(A_j(s;t) \in \{a_1, a_2, \cdots, a_j\}\)
through a piecewise function,

\begin{equation}
v[x_i(s;t)] = 
\begin{cases}
a_0 & C_1 \leq x^i(s; t) < C_0 \\
a_1 & C_2 \leq x^i(s; t) < C_1 \\
a_2 & C_3 \leq x^i(s; t) < C_2 \\
\cdots \\
a_z & C_z \leq x^i(s; t)
\end{cases}
\end{equation}

This is typically used at the end of the index pipeline to simplify the
index to communicate to the public the severity of the concept of
interest measured by the index. An example of simplification is to map
the calculated SPI to four categories: mild, moderate, severe, and
extreme drought.

\hypertarget{sec-software}{%
\section{Software design}\label{sec-software}}

The R package \texttt{tidyindex} implements a proof-of-concept of the
index pipeline modules described in Section~\ref{sec-pipeline}. These
modules compute an index in a sequential manner, as shown below:

\begin{Shaded}
\begin{Highlighting}[]
\NormalTok{DATA }\SpecialCharTok{|\textgreater{}} 
  \FunctionTok{module1}\NormalTok{(...) }\SpecialCharTok{|\textgreater{}}
  \FunctionTok{module2}\NormalTok{(...) }\SpecialCharTok{|\textgreater{}}
  \FunctionTok{module3}\NormalTok{(...) }\SpecialCharTok{|\textgreater{}}
\NormalTok{  ...}
\end{Highlighting}
\end{Shaded}

Each module offers a variety of alternatives, each represented by a
distinct function. For example, within the
\texttt{dimension\_reduction()} module, three methods are available:
\texttt{aggregate\_linear()}, \texttt{aggregate\_geometrical()}, and
\texttt{manual\_input()} and they can be used as:

\begin{Shaded}
\begin{Highlighting}[]
\FunctionTok{dimension\_reduction}\NormalTok{(}\AttributeTok{V1 =} \FunctionTok{aggregate\_linear}\NormalTok{(...))}
\FunctionTok{dimension\_reduction}\NormalTok{(}\AttributeTok{V2 =} \FunctionTok{aggregate\_geometrical}\NormalTok{(...))}
\FunctionTok{dimension\_reduction}\NormalTok{(}\AttributeTok{V3 =} \FunctionTok{manual\_input}\NormalTok{(...))}
\end{Highlighting}
\end{Shaded}

Each method can be independently evaluated as a recipe, for example,

\begin{Shaded}
\begin{Highlighting}[]
\FunctionTok{manual\_input}\NormalTok{(}\SpecialCharTok{\textasciitilde{}}\NormalTok{x1 }\SpecialCharTok{+}\NormalTok{ x2)}
\end{Highlighting}
\end{Shaded}

takes a formula to combine the variables \texttt{x1} and \texttt{x2} and
return:

\begin{verbatim}
[1] "manual_input"
attr(,"formula")
[1] "x1 + x2"
attr(,"class")
[1] "dim_red"
\end{verbatim}

This recipe will then be evaluated in the pipeline module with data to
obtain numerical results. The package also offers wrapper functions that
combine multiple steps for specific indexes. For instance, the
\texttt{idx\_spi()} function bundles three steps (temporal aggregation,
distribution fit, and normalizing) into a single command, simplifying
the syntax for computation. Analysts are also encouraged to create
customized indexes from existing modules.

\begin{Shaded}
\begin{Highlighting}[]
\NormalTok{idx\_spi }\OtherTok{\textless{}{-}} \ControlFlowTok{function}\NormalTok{(...)\{}
\NormalTok{  DATA }\SpecialCharTok{|\textgreater{}} 
    \FunctionTok{aggregate}\NormalTok{(...) }\SpecialCharTok{|\textgreater{}}
    \FunctionTok{dist\_fit}\NormalTok{(...) }\SpecialCharTok{|\textgreater{}} 
    \FunctionTok{augment}\NormalTok{(...)}
\NormalTok{\}}
\end{Highlighting}
\end{Shaded}

(more changes) The accompanied package, tidyindex, is not intended to
offer an exhaustive implementation for all indexes across every domains.
Instead, it provides a realization of the pipeline framework proposed in
the paper. When adopting the pipeline approach to construct indexes,
analysts may consider developing software that can be readily deployed
in the cloud for production purposes.

\hypertarget{sec-examples}{%
\section{Examples}\label{sec-examples}}

This section uses the example of drought and social indexes to show the
analysis made possible with the index pipeline. The drought index
example computes two indexes (SPI and SPEI) with various time scales and
distributions simultaneously using the pipeline framework to understand
the flood and drought events in Queensland. The second example focuses
on the dimension reduction step in the Global Gender Gap Index to
explore how the changes in linear combination weights affect the index
values and country rankings.

\hypertarget{every-distribution-every-scale-every-index-all-at-once}{%
\subsection{Every distribution, every scale, every index all at
once}\label{every-distribution-every-scale-every-index-all-at-once}}

The state of Queensland in Australia frequently experiences natural
disaster events such as flood and drought, which can significantly
impact its agricultural industry. This example uses daily data from
Global Historical Climatology Network Daily (GHCND), aggregated into
monthly precipitation, to compute two drought indexes -- SPI and SPEI --
at various time scales and fitted distributions, for 29 stations in the
state of Queensland in Australia, spanning from January 1990 to April
2022. This example showcases the basic calculation of indexes with
different parameter specifications within the pipeline framework. The
dataset used in this example is available in the \texttt{tidyindex}
package as \texttt{queensland} and blow prints the first few rows of the
data:

\begin{verbatim}
# A tibble: 5 x 9
  id                ym  prcp  tmax  tmin  tavg  long   lat name       
  <chr>          <mth> <dbl> <dbl> <dbl> <dbl> <dbl> <dbl> <chr>      
1 ASN00029038 1990 Jan  1682  34.3  24.7  29.5  142. -15.5 KOWANYAMA ~
2 ASN00029038 1990 Feb   416  35.2  23.2  29.2  142. -15.5 KOWANYAMA ~
3 ASN00029038 1990 Mar  2026  32.5  23.6  28.0  142. -15.5 KOWANYAMA ~
4 ASN00029038 1990 Apr   597  32.9  17.7  25.3  142. -15.5 KOWANYAMA ~
5 ASN00029038 1990 May   244  31.8  20.1  25.9  142. -15.5 KOWANYAMA ~
\end{verbatim}

\begin{figure}

{\centering \includegraphics[width=1\textwidth,height=0.9\textheight]{figures/pipeline-spei.png}

}

\caption{\label{fig-spei}Index pipeline for two drought indexes: the
Standardized Precipitation Index (SPI) and the Standardized
Precipitation-Evapotranspiration Index (SPEI). Both indexes share
similar construction steps with SPEI having two steps additional steps
(variable transformation and dimension reduction) to convert temperature
into evapotranspiration and combine it with the precipitation series.}

\end{figure}

Figure~\ref{fig-spei} illustrates the pipeline steps of the two indexes.
The two indexes are similar with the distinct that SPEI involves two
additional steps -- variable transformation and dimension reduction --
prior to temporal processing. As introduced in
Section~\ref{sec-software}, wrapper functions are available for both
indexes as \texttt{idx\_spi()} and \texttt{idx\_spei()}, which allows
for the specification of different time scales and distributions for
fitting the aggregated series. In \texttt{tidyindex}, multiple indexes
can be calculated collectively using the function
\texttt{compute\_indexes()}. Both SPI and SPEI are calculated across
four time scales (6, 12, 24, and 36 months). The SPEI is fitted with two
distributions (log-logistic and general extreme value distribution) and
the gamma distribution is used for SPI:

\begin{Shaded}
\begin{Highlighting}[]
\NormalTok{.scale }\OtherTok{\textless{}{-}} \FunctionTok{c}\NormalTok{(}\DecValTok{6}\NormalTok{, }\DecValTok{12}\NormalTok{, }\DecValTok{24}\NormalTok{, }\DecValTok{36}\NormalTok{)}
\NormalTok{idx }\OtherTok{\textless{}{-}}\NormalTok{ queensland }\SpecialCharTok{\%\textgreater{}\%}
  \FunctionTok{init}\NormalTok{(}\AttributeTok{id =}\NormalTok{ id, }\AttributeTok{time =}\NormalTok{ ym) }\SpecialCharTok{\%\textgreater{}\%}
  \FunctionTok{compute\_indexes}\NormalTok{(}
    \AttributeTok{spei =} \FunctionTok{idx\_spei}\NormalTok{(}
      \AttributeTok{.pet\_method =} \StringTok{"thornthwaite"}\NormalTok{, }\AttributeTok{.tavg =}\NormalTok{ tavg, }\AttributeTok{.lat =}\NormalTok{ lat, }
      \AttributeTok{.scale =}\NormalTok{ .scale, }\AttributeTok{.dist =} \FunctionTok{c}\NormalTok{(}\FunctionTok{gev}\NormalTok{(), }\FunctionTok{loglogistic}\NormalTok{())),}
    \AttributeTok{spi =} \FunctionTok{idx\_spi}\NormalTok{(}\AttributeTok{.scale =}\NormalTok{ .scale)}
\NormalTok{  )}
\end{Highlighting}
\end{Shaded}

\begin{verbatim}
# A tibble: 128,586 x 19
   .idx  .period id             ym  prcp  tmax  tmin  tavg  long   lat
   <chr>   <dbl> <chr>       <mth> <dbl> <dbl> <dbl> <dbl> <dbl> <dbl>
 1 spei        6 ASN0002~ 1990 Jun   170  29.7  16.2  23.0  142. -15.5
 2 spei        6 ASN0002~ 1990 Jun   170  29.7  16.2  23.0  142. -15.5
 3 spei        6 ASN0002~ 1990 Jun     0  23.0  11.8  17.4  139. -20.7
 4 spei        6 ASN0002~ 1990 Jun     0  23.0  11.8  17.4  139. -20.7
 5 spei        6 ASN0003~ 1990 Jun   794  25.8  18.1  21.9  146. -16.9
 6 spei        6 ASN0003~ 1990 Jun   794  25.8  18.1  21.9  146. -16.9
 7 spei        6 ASN0003~ 1990 Jun   504  23.0  13.8  18.4  145. -17.1
 8 spei        6 ASN0003~ 1990 Jun   504  23.0  13.8  18.4  145. -17.1
 9 spei        6 ASN0003~ 1990 Jun  1970  23.9  16.4  20.2  146. -17.6
10 spei        6 ASN0003~ 1990 Jun  1970  23.9  16.4  20.2  146. -17.6
# i 128,576 more rows
# i 9 more variables: name <chr>, pet <dbl>, diff <dbl>,
#   .scale <dbl>, .agg <dbl>, .method <chr>, .fitted <dbl>,
#   .dist <chr>, .index <dbl>
\end{verbatim}

The output contains the original data, index values (\texttt{.index}),
parameters used (\texttt{.scale}, \texttt{.method}, and \texttt{.dist}),
and all the intermediate variables (\texttt{pet}, \texttt{.agg}, and
\texttt{.fitted}). This data can be visualized to investigate the
spatio-temporal distribution of the drought/flood events, as well as the
response of index values to different time scales and distribution
parameters at specific single locations.
Figure~\ref{fig-compute-spatial} and Figure~\ref{fig-compute-temporal}
exemplify two possibilities. Figure~\ref{fig-compute-spatial} presents
the spatial distribution of SPI during two periods: October 2010 to
March 2011 for the 2010/11 Queensland flood and October 2019 to March
2020 for the 2019 Australia drought, which contributes to the notorious
2019/20 bushfire. Figure~\ref{fig-compute-temporal} displays the
sensitivity of the SPEI series at the Texas post office to different
time scales and fitted distributions. Larger time scales produce a
smoother index across time, however, all time scales indicate an extreme
drought (corresponding to -2 in SPEI) in 2020, confirming the severity
of the drought across different time horizons. Moreover, the chosen
distribution has less influence on the index, with general extreme value
distribution tending to produce more extreme outcomes than log-logistic
distribution for the extreme events (index \textgreater{} 2 or
\textless-2).

\begin{figure}

{\centering \includegraphics{tidyindex_files/figure-pdf/fig-compute-spatial-1.pdf}

}

\caption{\label{fig-compute-spatial}Spatial distribution of Standardized
Precipitation Index (SPI-12) in Queensland, Australia during two major
flood and drought events: 2010/11 and 2019/20. The map shows a
continuous wet period during the 2010/11 flood period and a mitigated
drought situation, after its worst in 2019 December and 2020 January,
likely due to the increased rainfall in February from the meteorological
record.}

\end{figure}

\begin{figure}

{\centering \includegraphics{tidyindex_files/figure-pdf/fig-compute-temporal-1.pdf}

}

\caption{\label{fig-compute-temporal}Time series plot of Standardized
Precipitation-Evapotranspiration Index (SPEI) at the Texas post office
station (highlighted by a diamond shape in panel a). The SPEI is
calculated at four time scales (6, 12, 24, and 36 months) and fitted
with two distributions (Log Logistic and GEV). The dashed line at -2
represents the class ``extreme drought'' by the SPEI. A larger time
scale gives a smoother index series, while also taking longer to recover
from an extreme situation as seen in the 2019/20 drought period. The
SPEI values from the two distribution fit mostly agree, while GEV can
result in more extreme values, i.e.~in 1998 and 2020.}

\end{figure}

\hypertarget{does-a-minor-change-in-variable-weights-cause-a-tornado}{%
\subsection{Does a minor change in variable weights cause a
tornado?}\label{does-a-minor-change-in-variable-weights-cause-a-tornado}}

The Global Gender Gap Index (GGGI), published annually by the World
Economic Forum, measures gender parity by assessing relative gaps
between men and women in four key areas: Economic Participation and
Opportunity, Educational Attainment, Health and Survival, and Political
Empowerment (World Economic Forum 2023). The index, compiled from 14
variables expressed as female-to-male ratios, first aggregates these
variables into four dimensions through linear combinations. These
dimensions are then combined through another linear combination to form
the index. Figure~\ref{fig-pp-gggi} illustrates this pipeline
construction by applying the dimension reduction module twice on the
data to generate the index. Table~\ref{tbl-gggi-weights} presents the
variable weights (\texttt{V-weight}) and dimension weights
(\texttt{D-weight}) used in the two dimension reduction steps. In the
table, the variable weights are computed as the inverse of the standard
deviation of each variable, scaled to sum to 1. These weights ensure
that a one percentage point change in the standard deviation of each
variable contributes equally to the index. Dimension weights are equal
across the four dimensions and the last column, \texttt{weight},
multiples the variable and dimension weights to produce a single set of
weights.

\begin{figure}

{\centering \includegraphics[width=1\textwidth,height=0.9\textheight]{figures/pipeline-gggi.png}

}

\caption{\label{fig-pp-gggi}Index pipeline for the Global Gender Gap
Index (GGGI). The index is constructed as applying the module dimension
reduction twice on the data.}

\end{figure}

\hypertarget{tbl-gggi-weights}{}
\begin{longtable}[]{@{}
  >{\raggedright\arraybackslash}p{(\columnwidth - 8\tabcolsep) * \real{0.5455}}
  >{\raggedright\arraybackslash}p{(\columnwidth - 8\tabcolsep) * \real{0.1299}}
  >{\raggedleft\arraybackslash}p{(\columnwidth - 8\tabcolsep) * \real{0.1169}}
  >{\raggedleft\arraybackslash}p{(\columnwidth - 8\tabcolsep) * \real{0.1169}}
  >{\raggedleft\arraybackslash}p{(\columnwidth - 8\tabcolsep) * \real{0.0909}}@{}}
\caption{\label{tbl-gggi-weights}Weights used to compute the Global
Gender Gap Index}\tabularnewline
\toprule()
\begin{minipage}[b]{\linewidth}\raggedright
Variable
\end{minipage} & \begin{minipage}[b]{\linewidth}\raggedright
Dimension
\end{minipage} & \begin{minipage}[b]{\linewidth}\raggedleft
V-weight
\end{minipage} & \begin{minipage}[b]{\linewidth}\raggedleft
D-weight
\end{minipage} & \begin{minipage}[b]{\linewidth}\raggedleft
Weight
\end{minipage} \\
\midrule()
\endfirsthead
\toprule()
\begin{minipage}[b]{\linewidth}\raggedright
Variable
\end{minipage} & \begin{minipage}[b]{\linewidth}\raggedright
Dimension
\end{minipage} & \begin{minipage}[b]{\linewidth}\raggedleft
V-weight
\end{minipage} & \begin{minipage}[b]{\linewidth}\raggedleft
D-weight
\end{minipage} & \begin{minipage}[b]{\linewidth}\raggedleft
Weight
\end{minipage} \\
\midrule()
\endhead
Labour force participation & Economy & 0.199 & 0.25 & 0.050 \\
Wage equality for similar work & Economy & 0.310 & 0.25 & 0.078 \\
Estimated earned income & Economy & 0.221 & 0.25 & 0.055 \\
Legislators senior officials and managers & Economy & 0.149 & 0.25 &
0.037 \\
Professional and technical workers & Economy & 0.121 & 0.25 & 0.030 \\
Literacy rate & Education & 0.191 & 0.25 & 0.048 \\
Enrolment in primary education & Education & 0.459 & 0.25 & 0.115 \\
Enrolment in secondary education & Education & 0.230 & 0.25 & 0.058 \\
Enrolment in tertiary education & Education & 0.121 & 0.25 & 0.030 \\
Sex ratio at birth & Health & 0.693 & 0.25 & 0.173 \\
Healthy life expectancy & Health & 0.307 & 0.25 & 0.077 \\
Women in parliament & Politics & 0.310 & 0.25 & 0.078 \\
Women in ministerial positions & Politics & 0.247 & 0.25 & 0.062 \\
Years with female head of state & Politics & 0.443 & 0.25 & 0.111 \\
\bottomrule()
\end{longtable}

The 2023 GGGI data is available from the Global Gender Gap Report 2023
in the country's economy profile and can be accessed in R via the
\texttt{tidyindex} package as \texttt{gggi} and
Table~\ref{tbl-gggi-weights} as \texttt{gggi\_weights}. The index can be
reproduced with the package as:

\begin{Shaded}
\begin{Highlighting}[]
\NormalTok{gggi }\SpecialCharTok{\%\textgreater{}\%} 
  \FunctionTok{init}\NormalTok{(}\AttributeTok{id =}\NormalTok{ country) }\SpecialCharTok{\%\textgreater{}\%}
  \FunctionTok{add\_meta}\NormalTok{(gggi\_weights, }\AttributeTok{var\_col =}\NormalTok{ variable) }\SpecialCharTok{\%\textgreater{}\%} 
  \FunctionTok{dimension\_reduction}\NormalTok{(}
    \AttributeTok{index\_new =} \FunctionTok{aggregate\_linear}\NormalTok{(}
      \SpecialCharTok{\textasciitilde{}}\NormalTok{labour\_force\_participation}\SpecialCharTok{:}\NormalTok{years\_with\_female\_head\_of\_state,}
      \AttributeTok{weight =}\NormalTok{ weight)) }
\end{Highlighting}
\end{Shaded}

After initialzing the \texttt{gggi} object and attaching the
\texttt{gggi\_weights} as meta-data, a single linear combination step
within the dimension reduction module is applied to the 14 variables
(from column \texttt{labour\_force\_participation} to
\texttt{years\_with\_female\_head\_of\_state}), using the weight
specified in the \texttt{weight} column of the attached metadata. While
computing the index from the original 14 variables, it remains unclear
how the missing values are handled within the index, which impacts 68
out of the total 146 countries. However, after aggregating variables
into the four dimensions, where no missing values exist, the index is
reproducible for all the countries.

A linear combination can also be interpreted as a linear projection of
multivariate information with a weight vector. Projecting data from
higher to lower dimensions unavoidably leads to information loss and the
weights used require careful examination to understand their effects on
the index and implications for interpretation and decision-making. By
making slight adjustments to the weight vector, we can observe how index
values and country rankings change. For illustration, we select a set of
countries including:

\begin{enumerate}
\def\labelenumi{\arabic{enumi})}
\tightlist
\item
  Top-ranked countries with GGGI \textgreater{} 0.85,
\item
  Countries ranked between 57 and 62, with GGGI values from 0.72 to
  0.73, and
\item
  Low-ranked countries with GGGI \textless{} 0.6.
\end{enumerate}

We slightly vary the weight of the politics dimension from the original
0.25 while keeping the weights constant for the other three dimensions.
This process creates an animation showing the movement of index values
in response to changing weights. This visualization technique, which
presents a sequence of data projections, is referred to as a ``tour''.
The specific kind of tour that moves between the original and nearby
projections is known as a ``local tour''.

In Figure~\ref{fig-idx-tour}, six frames have been chosen from the
animation available at https://vimeo.com/847874016. When the weight of
politics is reduced (Frame 1 and 6 vs.~Frame 12), the difference in GGGI
values between the top Nordic countries and mid- or low-ranked countries
narrows, suggesting a smaller variation among countries in achieving
gender parity. Conversely, when the weight of politics increases (Frame
18, 24, and 29 vs.~Frame 12), nearly all the countries in the three
categories experience a decrease in GGGI values. A noticeable exception
to this trend is Bangladesh, where its index value moves in the same
direction as the politics weight. This leads to its index value being
almost similar to those of the top-ranked Nordic countries when the
politics dimension is assigned a weight of 0.52 in Frame 29.

In GGGI, the index value has a direct interpretation as the percentage
of the gender gap that has been closed. When countries exhibit closer or
consistent decreasing index values, it can be directly interpreted as a
reflection of a country's progress toward gender parity. Ideally, an
index should be robust against minor changes in its construction
components. This example provides researchers with means to observe
changes in the index resulting from variations in the weights used in
linear combinations. This approach can be applied broadly to different
sets of weights of interest, extending beyond the change in a single
dimension illustrated by the example.

\begin{figure}

{\centering \includegraphics{tidyindex_files/figure-pdf/fig-idx-tour-1.pdf}

}

\caption{\label{fig-idx-tour}Six frames selected to explore how varying
the weights of the politics dimension changes the index values and
country rankings in Global Gender Gap Index (GGGI). The top panel shows
the GGGI value against the country, ranked by its original index value
in Frame 12. The bottom panel displays the weight used to produce the
index values in the top panel, with each frame corresponds to a set of
weights. Countries selected includes 1) top-ranked countries with GGGI
\textgreater{} 0.85, 2) countries ranked between 57 and 62 with GGGI
from 0.72 to 0.73, and 3) low-ranked countries with GGGI \textless{}
0.6. Compared to Frame 12 where equal weigts are used for the four
dimension, a reduced weight in politics (Frame 1 and 6) shows narrower
gaps between top and mid- or lower- ranked countries, while an increase
in the politics weight (Frame 18, 24, and 29) leads to a systematic
decrease of GGGI values across all the countries, except for Bangladesh.
Full animation is available at https://vimeo.com/847874016.}

\end{figure}

TODO: increase font for variable names

\hypertarget{confidence-interval}{%
\subsection{Confidence interval}\label{confidence-interval}}

\begin{Shaded}
\begin{Highlighting}[]
\NormalTok{DATA }\SpecialCharTok{\%\textgreater{}\%} 
  \FunctionTok{aggregate}\NormalTok{(}\AttributeTok{.var =}\NormalTok{ prcp, }\AttributeTok{.scale =} \DecValTok{24}\NormalTok{) }\SpecialCharTok{\%\textgreater{}\%} 
  \FunctionTok{dist\_fit}\NormalTok{(}\AttributeTok{.dist =} \FunctionTok{gamma}\NormalTok{(), }\AttributeTok{.var =}\NormalTok{ .agg, }\AttributeTok{.n\_boot =} \DecValTok{100}\NormalTok{) }\SpecialCharTok{\%\textgreater{}\%} 
  \FunctionTok{augment}\NormalTok{(}\AttributeTok{.var =}\NormalTok{ .agg)}
\end{Highlighting}
\end{Shaded}

Figure~\ref{fig-conf-interval} presents the 80\% and 95\% confidence
interval for the Texas post office station, in Queensland, Australia.
Since the start of 2019, both confidence intervals indicate an extreme
drought condition (below -2). early warning for the bushfire?

also the ABS one - could display some uncertainties

\begin{figure}

{\centering \includegraphics{tidyindex_files/figure-pdf/fig-conf-interval-1.pdf}

}

\caption{\label{fig-conf-interval}80\% and 95\% confidence interval of
the Standardized Precipitation Index (SPI) for the Texas post office
station, in Queensland, Australia. A bootstrap sample of 100 is taken
from the aggregated pricipitation series to estimate gamma parameters
and calculate the index. Since the start of 2019, both confidence
intervals indicate an extreme drought condition (below -2).}

\end{figure}

\hypertarget{conclusion}{%
\section{Conclusion}\label{conclusion}}

The paper introduces a data pipeline comprising nine modules designed
for the construction and analysis of indexes within the tidy framework.
The pipeline offers a modular workflow to allow compute index with
different parameterizations, to test minor changes to the original index
definition, and to quantify uncertainties. The framework proposed in the
paper is universal to index across diverse domains. Examples are
provided, including the drought indexes (SPI and SPEI) and Global Gender
Gap Index (GGGI), to demonstrate the index calculation with different
time scales and distributions and to illustrate how slight adjustment of
linear combination weights impact the index.

\hypertarget{acknowledgement}{%
\section{Acknowledgement}\label{acknowledgement}}

This work is funded by a Commonwealth Scientific and Industrial Research
Organisation (CSIRO) Data61 Scholarship. The article is created using
Quarto (Allaire et al. 2022) in R (R Core Team 2021). The source code
for reproducing this paper can be found at:
https://github.com/huizezhang-sherry/paper-tidyindex.

\hypertarget{reference}{%
\section*{Reference}\label{reference}}
\addcontentsline{toc}{section}{Reference}

\hypertarget{refs}{}
\begin{CSLReferences}{1}{0}
\leavevmode\vadjust pre{\hypertarget{ref-alahacoon_comprehensive_2022}{}}%
Alahacoon, Niranga, and Mahesh Edirisinghe. 2022. {``A Comprehensive
Assessment of Remote Sensing and Traditional Based Drought Monitoring
Indices at Global and Regional Scale.''} \emph{Geomatics, Natural
Hazards and Risk} 13 (December): 762--99.
\url{https://doi.org/10.1080/19475705.2022.2044394}.

\leavevmode\vadjust pre{\hypertarget{ref-Allaire_Quarto_2022}{}}%
Allaire, J. J., Charles Teague, Carlos Scheidegger, Yihui Xie, and
Christophe Dervieux. 2022. \emph{{Quarto}} (version 1.2).
\url{https://doi.org/10.5281/zenodo.5960048}.

\leavevmode\vadjust pre{\hypertarget{ref-COINr}{}}%
Becker, William, Giulio Caperna, Maria Del Sorbo, Hedvig Norlen, Eleni
Papadimitriou, and Michaela Saisana. 2022. {``COINr: An r Package for
Developing Composite Indicators.''} \emph{Journal of Open Source
Software} 7 (78): 4567. \url{https://doi.org/10.21105/joss.04567}.

\leavevmode\vadjust pre{\hypertarget{ref-SPEI}{}}%
Beguería, Santiago, and Sergio M. Vicente-Serrano. 2017. \emph{SPEI:
Calculation of the Standardised Precipitation-Evapotranspiration Index}.
\url{https://CRAN.R-project.org/package=SPEI}.

\leavevmode\vadjust pre{\hypertarget{ref-buja_elements_1988}{}}%
Buja, A, D Asimov, C Hurley, and JA McDonald. 1988. {``Elements of a
Viewing Pipeline for Data Analysis.''} In \emph{Dynamic Graphics for
Statistics}, 277--308. Wadsworth, Belmont.

\leavevmode\vadjust pre{\hypertarget{ref-fundiversity}{}}%
Grenié, Matthias, and Hugo Gruson. 2023. \emph{{fundiversity}: Easy
Computation of Functional Diversity Indices}.
\url{https://doi.org/10.5281/zenodo.4761754}.

\leavevmode\vadjust pre{\hypertarget{ref-hao_drought_2015}{}}%
Hao, Zengchao, and Vijay P. Singh. 2015. {``Drought Characterization
from a Multivariate Perspective: {A} Review.''} \emph{Journal of
Hydrology} 527 (August): 668--78.
\url{https://doi.org/10.1016/j.jhydrol.2015.05.031}.

\leavevmode\vadjust pre{\hypertarget{ref-jones_vulnerability_2007}{}}%
Jones, Brenda, and Jean Andrey. 2007. {``Vulnerability Index
Construction: Methodological Choices and Their Influence on Identifying
Vulnerable Neighbourhoods.''} \emph{International Journal of Emergency
Management} 4 (2): 269--95.
\url{https://doi.org/10.1504/IJEM.2007.013994}.

\leavevmode\vadjust pre{\hypertarget{ref-tidymodels}{}}%
Kuhn, Max, and Hadley Wickham. 2020. \emph{Tidymodels: A Collection of
Packages for Modeling and Machine Learning Using Tidyverse Principles.}
\url{https://www.tidymodels.org}.

\leavevmode\vadjust pre{\hypertarget{ref-gpindex}{}}%
Martin, Steve. 2023. \emph{Gpindex: Generalized Price and Quantity
Indexes}. \url{https://CRAN.R-project.org/package=gpindex}.

\leavevmode\vadjust pre{\hypertarget{ref-mckee1993relationship}{}}%
McKee, Thomas B, Nolan J Doesken, John Kleist, et al. 1993. {``The
Relationship of Drought Frequency and Duration to Time Scales.''} In
\emph{Proceedings of the 8th Conference on Applied Climatology},
17:179--83. 22. Boston, MA, USA.

\leavevmode\vadjust pre{\hypertarget{ref-oecd_handbook_2008}{}}%
OECD, European Union, and Joint Research Centre - European Commission.
2008. \emph{Handbook on {Constructing} {Composite} {Indicators}:
{Methodology} and {User} {Guide}}. OECD.
\url{https://doi.org/10.1787/9789264043466-en}.

\leavevmode\vadjust pre{\hypertarget{ref-R}{}}%
R Core Team. 2021. \emph{R: A Language and Environment for Statistical
Computing}. Vienna, Austria: R Foundation for Statistical Computing.
\url{https://www.R-project.org/}.

\leavevmode\vadjust pre{\hypertarget{ref-uncertainty}{}}%
Saisana, M., A. Saltelli, and S. Tarantola. 2005. {``{Uncertainty and
Sensitivity Analysis Techniques as Tools for the Quality Assessment of
Composite Indicators}.''} \emph{Journal of the Royal Statistical Society
Series A: Statistics in Society} 168 (2): 307--23.
\url{https://doi.org/10.1111/j.1467-985X.2005.00350.x}.

\leavevmode\vadjust pre{\hypertarget{ref-sutherland_orca_2000}{}}%
Sutherland, Peter, Anthony Rossini, Thomas Lumley, Nicholas Lewin-Koh,
Julie Dickerson, Zach Cox, and Dianne Cook. 2000. {``Orca: {A}
{Visualization} {Toolkit} for {High}-{Dimensional} {Data}.''}
\emph{Journal of Computational and Graphical Statistics} 9 (3): 509--29.
\url{https://www.jstor.org/stable/1390943}.

\leavevmode\vadjust pre{\hypertarget{ref-svoboda2016handbook}{}}%
Svoboda, Mark, Brian Fuchs, et al. 2016. {``Handbook of Drought
Indicators and Indices.''} \emph{Drought and Water Crises: Integrating
Science, Management, and Policy}, 155--208.

\leavevmode\vadjust pre{\hypertarget{ref-tate_social_2012}{}}%
Tate, Eric. 2012. {``Social Vulnerability Indices: A Comparative
Assessment Using Uncertainty and Sensitivity Analysis.''} \emph{Natural
Hazards} 63 (2): 325--47.
\url{https://doi.org/10.1007/s11069-012-0152-2}.

\leavevmode\vadjust pre{\hypertarget{ref-spei}{}}%
Vicente-Serrano, Sergio M., Santiago Beguería, and Juan I. López-Moreno.
2010. {``A {Multiscalar} {Drought} {Index} {Sensitive} to {Global}
{Warming}: {The} {Standardized} {Precipitation} {Evapotranspiration}
{Index}.''} \emph{Journal of Climate} 23 (7): 1696--1718.
\url{https://journals.ametsoc.org/view/journals/clim/23/7/2009jcli2909.1.xml}.

\leavevmode\vadjust pre{\hypertarget{ref-wickham_tidy_2014}{}}%
Wickham, Hadley. 2014. {``Tidy {Data}.''} \emph{Journal of Statistical
Software} 59 (September): 1--23.
\url{https://doi.org/10.18637/jss.v059.i10}.

\leavevmode\vadjust pre{\hypertarget{ref-wickham_welcome_2019}{}}%
Wickham, Hadley, Mara Averick, Jennifer Bryan, Winston Chang, Lucy
D'Agostino McGowan, Romain François, Garrett Grolemund, et al. 2019.
{``Welcome to the {Tidyverse}.''} \emph{Journal of Open Source Software}
4 (43): 1686. \url{https://doi.org/10.21105/joss.01686}.

\leavevmode\vadjust pre{\hypertarget{ref-wickham_plumbing_2009}{}}%
Wickham, Hadley, Michael Lawrence, Dianne Cook, Andreas Buja, Heike
Hofmann, and Deborah F. Swayne. 2009. {``The Plumbing of Interactive
Graphics.''} \emph{Computational Statistics} 24 (2): 207--15.
\url{https://doi.org/10.1007/s00180-008-0116-x}.

\leavevmode\vadjust pre{\hypertarget{ref-WEF2023}{}}%
World Economic Forum. 2023. {``{The Global Gender Gap Report 2023}.''}
\url{https://www3.weforum.org/docs/WEF_GGGR_2023.pdf}.

\leavevmode\vadjust pre{\hypertarget{ref-xie_reactive_2014}{}}%
Xie, Yihui, Heike Hofmann, and Xiaoyue Cheng. 2014. {``Reactive
{Programming} for {Interactive} {Graphics}.''} \emph{Statistical
Science} 29 (2): 201--13.
\url{https://www.jstor.org/stable/43288470?seq=1}.

\leavevmode\vadjust pre{\hypertarget{ref-zargar2011review}{}}%
Zargar, Amin, Rehan Sadiq, Bahman Naser, and Faisal I Khan. 2011. {``A
Review of Drought Indices.''} \emph{Environmental Reviews} 19 (NA):
333--49. \url{https://www.jstor.org/stable/envirevi.19.333}.

\end{CSLReferences}



\end{document}
